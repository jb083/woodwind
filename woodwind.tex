\documentclass[a4paper,uplatex,dvipdfmx]{jsarticle}
\usepackage{amsmath,amssymb,mathrsfs,bm,braket}
% \usepackage{amsthm,amscd} %定理環境, 可換図式
% \usepackage{ascmac} %screen, itembox環境
% \usepackage{graphicx,xcolor}
\usepackage[dvipdfmx]{graphicx,xcolor}
\usepackage{fancyhdr,lastpage} %ヘッダー/フッター操作
% \usepackage{makeidx} %索引
\usepackage{hyperref} %ハイパーリンク
% \hypersetup{colorlinks=true,linkcolor=blue,citecolor=green}

% jsarticle用
\renewcommand{\abstractname}{Abstract}
\renewcommand{\figurename}{Fig.~}
\renewcommand{\tablename}{Table }

% 高さの設定
\setlength{\textheight}{\paperheight}   % ひとまず紙面を本文領域に
\setlength{\topmargin}{-5.4truemm}      % 上の余白を20mm(=1inch-5.4mm)に
\addtolength{\topmargin}{-\headheight}  %
\addtolength{\topmargin}{-\headsep}     % ヘッダの分だけ本文領域を移動させる
\addtolength{\textheight}{-4.4truecm}    % 下の余白も20mmに

\pagestyle{fancy}
	\lhead{}
	\chead{}
	\rhead{}
	\lfoot{}
	\cfoot{--\ \thepage\ /\ \pageref*{LastPage}\ --}
	\rfoot{}
	\renewcommand{\footrulewidth}{0.0pt} %デフォルト0.4pt
	\renewcommand{\headrulewidth}{0.0pt} %デフォルト0.4pt

\renewcommand{\r}{\bm{r}}
\newcommand{\x}{\bm{x}}
\renewcommand{\v}{\bm{v}}
\newcommand{\p}{\bm{p}}
\renewcommand{\k}{\bm{k}}
\renewcommand{\u}{\bm{u}}
\newcommand{\n}{\textbf{n}}

\title{木管楽器の発音原理に関するノート}
\author{H.~S.}
\date{2017.5.18-}

\begin{document}

\maketitle
\thispagestyle{fancy}

\begin{abstract}
	本記事は木管楽器の発音原理に関するノートである.
	特に, 管内での固有振動モードの解析的導出がメインである.
\end{abstract}

\tableofcontents

\section{疎密波の方程式}

\subsection{Euler方程式}

音波の発生および伝播を議論する場合, 空気 (水でもよいが) の粘性は無視できる.
また, 熱伝導も無視できるので, すべての運動は断熱的であり, 各流体要素の比エントロピー (単位質量あたりのエントロピー) は保存する.
\begin{align*}
	s = \mathrm{const.}
\end{align*}
従って音響学の範疇では媒質の状態は密度$\rho$および流体速度$\v$の$4$自由度によって記述される.

媒質の運動は連続方程式
\begin{align}
	\frac{ \partial \rho }{ \partial t } + \v \cdot \nabla \rho = - \rho \nabla \cdot \v
\end{align}
およびEuler方程式
\begin{align}
	\frac{ \partial \v }{ \partial t } + ( \v \cdot \nabla ) \v = - \frac{ 1 }{ \rho } \nabla p
\end{align}
によって与えられる. $p$は流体の圧力であり, 流体運動が断熱的であるという仮定のもとでは密度$\rho$だけの関数である: $p = p ( \rho )$.

理想気体の状態方程式が成り立つならば, 圧力$p$は断熱指数 (比熱比) $\gamma$を用いて
\begin{align}
	p = p_0 \left( \frac{ \rho }{ \rho_0 } \right)^\gamma \label{eos}
\end{align}
と表示できる. $\rho_0$, $p_0$は流体静止時 (つまり音波がないとき) の密度, 圧力である.
\ref{sec: wave equation}節で示すように流体中を伝播する音波の速度 (音速) $c_s$は
\begin{align}
	c_s = \sqrt{ \frac{ d p }{ d \rho } }
\end{align}
で与えられるので, 式(\ref{eos})を代入すると
\begin{align}
	c_s^2 = \gamma \frac{ p_0 }{ \rho_0 } \left( \frac{ \rho }{ \rho_0 } \right)^\gamma
\end{align}
となる. 特に$\rho = \rho_0$で評価すると$c_s^2 |_0 = \gamma p_0 / \rho_0$である.


\subsection{波動方程式}\label{sec: wave equation}

密度$\rho_0$, 圧力$p_0$の静的な配位にわずかな擾乱が加わるという状況を考える.
この場合, 位置$\x$における時刻$t$の密度$\rho ( t, \x )$の代わりに, 無次元化された密度ゆらぎ$\delta$および無次元速度$\u$を用いるのが便利である. その定義は
\begin{align}
	\delta ( t, \x ) := \frac{ \rho ( t, \x ) - \rho_0 }{ \rho_0 } , \ \ \u ( t, \x ) = \frac{ \v ( t, \x ) }{ c_s }
\end{align}
であり (以下$c_s |_0$を単に$c_s$と書く), いま$\delta$および$\u$は$1$よりずっと小さな量なので, これらの量について$2$次以上の項は無視してよい.
その近似では連続方程式およびEuler方程式は
\begin{align}
	\frac{ \partial \delta }{ \partial t } + c_s \nabla \cdot \u = 0 , \ \
	\frac{ \partial \u }{ \partial t } + c_s \nabla \delta = 0
\end{align}
となり, 速度$\u$を消去すると波動方程式
\begin{align}
	\left( \frac{ \partial^2 }{ \partial t^2} - c_s^2 \nabla^2 \right) \delta = 0
\end{align}
に到達する.


\section{閉管の固有振動}

\subsection{方形管の固有振動}

\subsection{円形管の固有振動}

\subsection{円錐管の固有振動}


\section{開菅と半開菅の固有振動}

\subsection{開菅の境界条件}

\subsection{方形開菅の固有振動}

\subsection{円形開菅の固有振動}

\subsection{円形半開菅の固有振動}

\subsection{円錐開管の固有振動}


\section{管外への音の放射}


\section{トーンホールの効果}

\end{document}
